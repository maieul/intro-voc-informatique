\section[Communiquer]{Communiquer entre ordinateurs} % Prendre le modèle TCP/IP en 4 couches, qu'on simplifie en 3 le but n'est pas de faire un cours sur le sujet https://fr.wikipedia.org/wiki/Suite_des_protocoles_Internet



\subsection{Le modèle des couches}
\begin{frame}
	\begin{itemize}
		\item Un réseau informatique est ensemble d'équipement reliés entre eux pour échanger de l'information.
		\item Le modèle \textbf{\formeenglish{Open System Interconnexion}} propose un standard en \textbf{7 couches}.
		\item Mais on simplifiera ici à \textbf{3 couches}. % FIXME lesquelles ? et elles devraient apparaître explicitement par la suite
		\begin{itemize}
			\item<.->Transmission physique de l'information.
			\item<.->Acheminement au bon destinataire.
			\item<.->Contenu et structure des échanges.
		\end{itemize}		
	\end{itemize}
\end{frame}
\subsection{Transmission physique de l'information}

\begin{frame}
	\begin{itemize}
		\item Deux humains qui communiquent utilise un support physique pour le faire :
 			\begin{itemize}
				\item Son. 
				\item Signaux de fumées.
				\item Gestes.
				\item etc.
			\end{itemize}
		\item Les ordinateurs de même:
			\begin{itemize}
				\item Câbles métalliques, notamment cuivres.
				\item Ondes électromagnétiques (Wifi, 3G etc.).
				\item Fibre optique. % Note : la lumière est une onde élèctromagnétique de fréquence particulière, mais passons.
			\end{itemize} 
		\item Il existe des \textbf{normes} décrivant l'usage de ces supports.
		\item Notamment les normes \textbf{ethernet}.
	\end{itemize}
\end{frame}


\subsection{Acheminement au bon destinataire}

\begin{frame}
	\begin{itemize}
		\item Pour s'avoir à qui on écrit on utilise une adresse.
		\item La poste a des règles internes d'acheminement du courrier.
		\item Le système le plus répandu aujourd'hui est \textbf{Internet Protocol} (1980).
	\end{itemize}
\end{frame}

\begin{frame}
	\begin{itemize}
		\item On peut utiliser IP sans Internet : réseaux locaux.
		\item L'interconnexion des \textbf{réseaux locaux} forme \textbf{Internet}.
		\item IP se base sur notion de \textbf{paquets}:
			\begin{itemize}
				\item On divise les données en paquets.
				\item On ne sait pas d'avance par où passent les paquets.
				\item Se caractérise notamment par l'aspect \textbf{décentralisé}.
			\end{itemize}
		\item \formeenglish{Internet Corporation for Assigned Names and Numbers} attribue les adresses IP sur Internet:
			\begin{itemize}
				\item IPv4 : \num{4,3 94 967 296} : épuisées.
				\item IPv6 : \num{3,4e38} : \SI{3}{\percent} des adresses utilisées?
			\end{itemize}
	\end{itemize}

\end{frame}

\begin{frame}
	\begin{itemize}
		\item Pour éviter de retenir les adresses IP on a inventé (1983) le \textbf{nom de domaine}.
		\item Les domaines fonctionnent de manière hiérarchiques.
		\begin{description}
		  \item[Premier niveau (TLD\footnote{\formeenglish{Top Level Domain.})] : extensions (\url{.fr}, \url{.org}). % FIXME nombre limité, gérés par des entités (AFNIC…) imposant des règles -> explique .name, .org, .fr…
			\item[Second niveau] : domaines (\url{wikipedia.org}, \url{unil.ch}). % FIXME loué par l'éditeur du site
			\item[Troisième niveau] : sous-domaine (\url{fr.wikipedia.org}). % sous le contrôle du site
			\item etc.
		\end{description}
	\end{itemize}
\end{frame}

\subsection{Contenu et structure des échanges} 

\begin{frame}
	\begin{itemize}
		\item Courriers électroniques.
		\item Échanges de fichiers.
		\item Messageries instantanées.
		\item Le plus fameux \textbf{le World Wide Web/la Toile} (1990).
	\end{itemize}
\end{frame}


\begin{frame}
	\begin{itemize}
		\item Mettre à disposition de l'information.
		\item Relier de l'information sur la base de liens \textbf{hypertextes}.
		\item Pas de système centralisé.
	\end{itemize}
\end{frame}

\begin{frame}
	\begin{itemize}
		\item Une page web est envoyée à un \textbf{client} web par un \textbf{serveur} via \formeenglish{HyperText Transfer Protocol} (HTTP).
		\item Tout ordinateur est un serveur potentiel.
		\item Des \textbf{hébergeurs} proposent des  ordinateurs spécialisés comment serveur HTTP.
	\end{itemize}
\end{frame}

\begin{frame}
	Une page web :
	\begin{itemize}
		\item Localisable par \formeenglish{Uniform Resource Locator} (URL).
		\item Décrite en \formeenglish{HyperText Markup Language} (HTML).
		\item Mise en forme via \formeenglish{Cascading Style Sheets} (CSS).
		\item Complétée par du multimédia.
		\item Animée via \formeenglish{JavaScript}. % FIXME en fait tu peux dire qu'une page web suffisemment complexe est un logiciel à part entière
	\end{itemize}

\end{frame}

\begin{frame}
	Deux types : % est-ce bien important ?
	\begin{description}
		\item[Statiques] : pages produites sur l'ordinateur du webmestre.
		\item[Dynamiques] : pages produites sur le serveur. Facilite la gestion de contenu éditorial
		\begin{itemize}
			\item Logiciel de forums.
			\item Logiciel de Wiki.
			\item Logiciels de gestion de contenu (SPIP, WordPress, Drupal).
			\item etc. % mes parents voient pas la distinction entre web et mail, puisqu'ils utilisent GMail. Ils voient pas l'intérêt d'un client mail natif, par exemple
		\end{itemize} 
	\end{description}
\end{frame}
