\section[Décrire]{Décrire les données}
\subsection{Des nombres}

\begin{slide}
	\begin{itemize}
		\item Aujourd'hui les données que manipulent la machine sont \textbf{des nombres}.
		\item Manipulées sous forme \textbf{binaire} pour des contraintes matérielles:
			\begin{itemize}
				\item Le courant passe ou ne passe pas.
				\item Le métal est chargé ou non chargé.
				\item Le plastique est gravé ou non gravé.
			\end{itemize}  
	\end{itemize}
\end{slide}

\begin{slide}
\begin{exampleblock}{Notation décimale}
\begin{align*}
	% Première ligne d'exemple
	\puissance{3}{4}%
	\puissance{2}{3}%
	\puissance{1}{2}%
	\puissance{0}{5} %
	&= 
	\puissance{3}{4000} &+&\  
	\puissance{2}{300} &+&\  
	\puissance{1}{20} &+&\  
	\puissance{0}{5}\\
	%
	% Seconde ligne d'exemple
	\only<+->{%
		&=
		\puissance{3}{4 \times 1000}
		&+&\  
		\puissance{2}{3 \times 100} 
		&+&\  
		\puissance{1}{2 \times 10} 
		&+&\  
		\puissance{0}{5 \times 1}
		} 
		\\
	%
	% Troisième ligne d'exemple
    \only<+->{%
    	&= 
		\puissance{3}{4 \times 10^3} 
		&+&\  
		\puissance{2}{3 \times 10^2} 
		&+&\  
		\puissance{1}{2 \times 10^1} 
		&+&\  
		\puissance{0}{5 \times 10^0}
		}
		\\
\end{align*}

\end{exampleblock}

\begin{exampleblock}{Notation binaire}
\begin{align*}
	% Première ligne d'exemple
	\overline{%
		\puissance{3}{1}%
		\puissance{2}{0}%
		\puissance{1}{1}%
		\puissance{0}{0}%
	}^2%	
	&= 
	\puissance{3}{1 \times 2^3} 
	&+&\  
	\puissance{2}{0 \times 2^2} 
	&+&\  
	\puissance{1}{1 \times 2^1} 
	&+&\  
	\puissance{0}{0 \times 2^0} 
	\\
	%Seconde ligne d'exemple
	\only<+->{%
		&= 
		\puissance{3}{1 \times 8} 
		&+&\  
		\puissance{2}{0 \times 4} 
		&+&\  
		\puissance{1}{1 \times 2} 
		&+&\  
		\puissance{0}{0 \times 1}
		}
		\\	
	% Troisième ligne d'exemple
	\only<+->{%
		&= 
		\puissance{3}{8} 
		&+&\  
		\puissance{2}{0} 
		&+&\  
		\puissance{1}{2} 
		&+&\  
		\puissance{0}{0}
		}
		\\
	% Quatrième ligne d'exemple																	
	&\only<+->{= 10}
\end{align*}
\end{exampleblock}
\end{slide}

\begin{slide}
	\begin{itemize}
		\item 1 unité minimale d'information (0/1) = 1 \textbf{bit}.
		\item 8 \textbf{bits} forme 1 \textbf{octet}.
		\item 1000~\textbf{octets} forment 1~\textbf{ko}.
		\item … bien que certains logiciels utilisent encore 1 ko = 1024 octets.
	\end{itemize}
\end{slide}

% FIXME: par ici tu pourrais montrer un exemple de calcul mécanique comme http://woodgears.ca/marbleadd/index.html
% cela illustre que le binaire est une représentation qui permet de faire des calculs mathématiques avec un objet matériel
% il y a des exemples de portes logiques hydrauliques aussi
\begin{slide}
	\begin{itemize}
		\item La \textbf{machine} manipule des \textbf{nombres}.
		\item L'\textbf{informaticien} manipule lui d'autres \textbf{types de données}:
		\begin{itemize}
			\item Nombres.
			\item Chaînes de caractères.
			\item Tableaux de données.
			\item Images.
			\item etc. 
			\item Chaque langage de programmation propose ses propres types de données.
		\end{itemize}
	\end{itemize}
\end{slide}

\subsection{De l'ASCII à l'Unicode}
% FIXME: je ne sais pas si ça apporte grand chose de parler d'unicode sans donner d'exemple
% sinon UTF ou ascii ça reste un truc technologique abstrait
% tu devrais montrer comment on représente un mot par un nombre (en montrant la table ascii et le codage du mot)
% puis poser la question "oui mais les accents ? le chinois ?" et introduire les autres codages puis unicode en disant juste que c'est une table bien plus grande
% montrer l'encodage d'un mot en unicode et expliquer le problème typique quand on interprète un nombre avec la mauvaise table
\begin{slide}
	\begin{itemize}
		\item Manipuler des chaînes de caractères est un des usages courants de l'informatique.
		\item D'autant plus qu'il est plus facile pour un humain d'utiliser des mots que des nombres pour :
			\begin{itemize}
				\item Représenter les données.
				\item Représenter les instructions.
			\end{itemize}
	\end{itemize}
\end{slide}
\begin{slide}
	\begin{itemize}
		\item Définir une liste de caractères : \textbf{jeu de caractères}.
		\item Attribuer un nombre différent à chaque caractère : \textbf{jeu de caractères codés}.
		\item Définir comment on stocke ces caractères : \textbf{codages des caractères}.
	\end{itemize}
\end{slide}

\begin{slide}
	\begin{itemize}
		\item \textbf{ASCII} (1963), \formeenglish{American Standard Code for Information Interchange} : \textbf{codage de caractères}:
			\begin{itemize}
				\item Met fin à la multiplication des précédents codages.
				\item Mais ne permet que de noter l'américain et les caractères informatiques de base (pas d'accents).
			\end{itemize}
		\item Multiplication des encodages pour les langues non anglo-saxonnes.
		\item \textbf{Unicode} (1993) : \textbf{jeu de caractères codés}:
			\begin{itemize}
				\item Vise à couvrir tous les caractères anciens et modernes, y compris ceux non encore existants.
				\item Peut s'implémenter sous différents \textbf{codages} : UTF-8, UTF-16, UTF-32.
			\end{itemize}
	\end{itemize}
\end{slide}

\subsection{De la structuration}
\begin{slide}
	\begin{itemize}
		\item Acte n°1 de tout informaticien : penser la structure des données.
		\item Exemple : faire une liste des ingrédients d'une recette de cuisine.
	\end{itemize}
\end{slide}

\begin{slide}

\begin{columns}[T]
	\begin{column}{0.25\textwidth}<.->
		\beamerdefaultoverlayspecification{}
		\begin{exampleblock}{En vrac}
		3 kilos de patates ; 5 choux rouges ; radis : une botte
		\end{exampleblock}	
	\end{column}
	
	\begin{column}{0.25\textwidth}
		\beamerdefaultoverlayspecification{}
		\begin{exampleblock}{Ingrédient/Qt.}
			\begin{itemize}
				\item Patates : 3~kg.
				\item Choux rouges : 5.
				\item Radis : 1~botte.
			\end{itemize}
		\end{exampleblock}
	\end{column}
	
	\begin{column}{0.25\textwidth}
		\beamerdefaultoverlayspecification{}
		\begin{exampleblock}{Ing./Unit./Qt.}
			\begin{itemize}
				\item Patates / 3 / kg
				\item Choux rouges / 5 / $\varnothing$\
				\item Radis / 1 / botte
			\end{itemize}
		\end{exampleblock}
	\end{column}
	\begin{column}{0.25\textwidth}
		\beamerdefaultoverlayspecification{}
		\begin{exampleblock}{Avec deux listes}
			\begin{itemize}
				\item Patates : 3
				\item Choux rouges : 5
				\item Radis : 1
			\end{itemize}
			\hrule width \textwidth
			\begin{itemize}
				\item Patates : kg
				\item Choux rouges : $\varnothing$\
				\item Radis : botte
			\end{itemize}
		\end{exampleblock}
	\end{column}
\end{columns}
\end{slide}

\subsubsection{Des langages de descriptions}

\begin{slide}
	\begin{itemize}
		\item Une fois conçu la \textbf{structure des données} l'informaticien va pouvoir les \textbf{encoder} (ou les faire encoder…).
		\item Il va pour cela utiliser un \textbf{langage de description}.
		\item Qu'il choisira en fonction de critères tels que:
			\begin{itemize}
				\item Les outils disponibles pour manipuler ces langages.
				\item La verbosité du langage.
				\item La diffusion du langage.
				\item L'inter-opérabilité qu'il souhaite avec d'autres outils.
				\item etc.
				\item Sa propre formation et ses a-prioris.
			\end{itemize}
	\end{itemize}
\end{slide}

\begin{slide}

		\beamerdefaultoverlayspecification{}
		\only<+>{
		\begin{exampleblock}{Directement dans le langage de programmation (ici Python)}
			\inputminted{python}{exemples-structure/python.py}
		\end{exampleblock}
		}

		\only<+>{
		\begin{exampleblock}{En \forme{eXtensible Markup Language} (XML)}
			\inputminted{xml}{exemples-structure/xml.xml}
		\end{exampleblock}
		}
		
		\only<+>{
		\begin{exampleblock}{En \forme{YAML Ain't Markup Language}}
			\inputminted{yaml}{exemples-structure/yaml.yaml}
		\end{exampleblock}
		}
		
		\only<+>{
		\begin{exampleblock}{En \forme{JavaScript Object Notation} (JSON)}
			\inputminted{json}{exemples-structure/json.json}
		\end{exampleblock}
		}
\end{slide}

\subsubsection{Des bases de données relationelles}
\begin{slide}
	\begin{itemize}
		\item On peut mettre des \textbf{données} dans des \textbf{fichiers} qu'on décode ensuite.
		\item Mais on peut aussi \textbf{faire abstraction} des fichiers pour ne se focaliser que sur les \textbf{données} et leurs \textbf{relations}.
	\end{itemize}
\end{slide}

\begin{slide}
	\begin{itemize}
		\item On utilise alors un \textbf{gestionnaire de bases de données}.
		\item Qu'on manipule avec un \textbf{langage spécifique}.
		\item Langage commun et standardisé : \enquote{Structured Query Language} (SQL).
	\end{itemize}
\end{slide}

\begin{slide}
	\begin{itemize}
				\item Langage déclaratif.
				\item Pertinents dans certains contextes notamment quand :
				\begin{itemize}
					\item Les relations entres les données sont simples et peut variantes.
					\item Les relations entre les données ne sont pas principalement hiérarchiques.
					\item On cherche plus à obtenir des données en fonctions de leur relations qu'à effectuer des manipulations dessus.
					\item On connaît les contraintes des relations entre données (unicités…)
					\item On veut gérer facilement la modification de plusieurs données en même temps.
				\end{itemize}
	\end{itemize}
	
\end{slide}
\begin{slide}
	
	\beamerdefaultoverlayspecification{}
	\scriptsize
	\begin{exampleblock}{Tableaux}
	\only<+->{\begin{tabular}{|l|l|c|}
	\hline 
		\multicolumn{3}{|c|}{\titretableau{Ingredients\_recettes}}\\
	\hline
		\titrecolonne{Recette} & \titrecolonne{Ingredients} &  \titrecolonne{Quantite} 
		\\
	\hline
		Bougli-boulga 	&	Patates 		& 3 \\
		Bougli-boulga	&	Choux rouges 	& 5 \\
		Bougli-boulga	& 	Radis 			& 1 \\
	\hline
	\end{tabular}
	}


	\only<+->{
	\begin{tabular}{|l|l|}
	\hline 
		\multicolumn{2}{|c|}{\titretableau{Unites\_ingredients}}\\
	\hline
		\titrecolonne{Ingredient} 	& \titrecolonne{Unite} \\
	\hline
		Patates					& kg \\
		Choux rouges			& \~\ \\
		Radis					& botte \\
		\hline
	\end{tabular}
	}


	\only<+->{\begin{tabular}{|l|c|}
		\hline \multicolumn{2}{|c|}{\titretableau{Recettes}}\\
	\hline
		\titrecolonne{Nom} 
		& 
		\titrecolonne{Temps} 
		\\
	\hline
	Bougli-boulga & 5 mn\\
	\hline
	\end{tabular}
	}
	\end{exampleblock}
	\only<+->{
	\begin{exampleblock}{Requête}
		\inputminted{sql}{exemples-structure/sql.sql}
	\end{exampleblock}
	}
\end{slide}

