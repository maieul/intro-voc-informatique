\section[Manipuler]{Manipuler les données}
\subsection{Encore du binaire}

\begin{slide}
	\begin{itemize}
		\item[Années 1950] Premiers ordinateurs à transistors.
		\item[Années 1970] Premiers ordinateurs à microprocesseur. % Actuellement (2012, de l'ordre du milliard), gravé de l'ordre de de la dizaine de nanomètre, soit 10 000em de cheveux
	\end{itemize}
\end{slide}

\begin{slide}
	\begin{itemize}
		\item Le processeur traite les données.
		\item Il est composé de plusieurs transistors, avec des fonctions spécialisées.
		\item Il possède un jeu d'instruction spécifique.
		\item Chaque instruction est codée en binaire.
	\end{itemize}
\end{slide}

\subsection{Des langages de différents niveaux}

\begin{slide}
	\begin{itemize}
		\item Des langages de très bas niveau, dépendant du processeur:
		\begin{itemize}
			\item Binaire.
			\item Assembleur.
		\end{itemize}
		\item Des langages de bas niveau:
		\begin{itemize}
			\item Basic
			\item C (et ses dérivés [C++,C\# etc.]?)
			\item etc.
		\end{itemize}
		\item Des langages de haut niveau:
		\begin{itemize}
			\item Java
			\item PHP
			\item Python
			\item etc.
		\end{itemize}
	\end{itemize}
\end{slide}

\subsection{Des langages selon les besoins}
\begin{slide}
	\begin{itemize}
		\item Contexte d'utilisation.
		\item Type de données à manipuler.
		\item Performance.
		\item Simplicité et lisibilité du code.
		\item Portabilité.
		\item Habitude… 
	\end{itemize}
\end{slide}





\subsection{Précisions de vocabulaire}
\begin{slide}
	\small
	\begin{itemize}
		\small
	  \item[Algorithme] : description logique des opérations nécessaire au traitement de l'information.
		\item[Programme] : expression d'un algorithme sous forme compréhensible par un ordinateur.
		\item[Fonction ou routine] : sous-programme.
		\item[Logiciel] : suite de programmes et d'outils pour les faire fonctionner (images, réglages etc.).
		\item[Application] : logiciel utilisé pour réaliser une activité. % sous-entend un produit commercial ou non
		\item En pratique \textbf{programme}, \textbf{logiciel} et \textbf{application} sont souvent interchangeables.
	\end{itemize}

\end{slide}
