\section[Traiter]{Traiter l'information}

\subsection{Informatique}

\begin{slide}
	\begin{itemize}
	\item Tous les jours nous manipulons de l'information pour accomplir des tâches.
	\item Exemple : déterminer un trajet pour se rendre quelque part.
		
		\begin{itemize}
			\item Moyens de transport.
			\item Budget.
			\item Choix politiques et écologiques.
			\item Urgence.
			\item etc.
		\end{itemize}
	\item À partir de là on détermine le meilleur trajet.
	
	\end{itemize}
\end{slide}

\begin{slide}

Deux postulats de l'informatique:
	\begin{enumerate}
		\item On peut systématiser le traitement des données.
		\item On peut automatiser le traitement des données.
	\end{enumerate}
\end{slide}

\begin{slide}

	\note[item]{\item De ces deux idées, c'est la première la plus importante, qui définit ce qu'est l'informatique \textbf{scientifique}.}
	
	
	\begin{block}{Informatique}
	Science du traitement systématique de l'information.
	\end{block}
	
	\note[item]{\textbf{En pratique}, l'informatique utilise des machines pour traiter l'information mais:}
	
	\begin{exampleblock}{Citation apocryphe (?) attribuée à Edsger Dijkstra}% Voir https://en.wikiquote.org/wiki/Computer_science#Disputed pour l'attribution
        \it
	Computer Science is no more about computers \\
        than astronomy is about telescopes.\\
	
        \medskip
	L'informatique n'est pas plus la science des ordinateurs que l'astronomie n'est celle des télescopes.
	\end{exampleblock}
	
\end{slide}

\begin{slide}
	\begin{block}{Informaticien (sens strict)}<.->
		\begin{itemize}
			\item Conçoit comment structurer des données.
			\item Conçoit comment les manipuler de manière systématique : définit des \textbf{algorithmes}.
		\end{itemize}
	\end{block}
\end{slide}
\subsection{Numérique et \enquote{digital}}

\begin{slide}

	\begin{itemize}
		\item Pour permettre l'automatisation du traitement des données par des machines on les ramène à des \textbf{nombres}.
		\item D'où :
			\begin{itemize}
				\item \forme{Numérique}
				\item \forme{Digital} (\formeenglish{digit} = \forme{chiffre})
			\end{itemize}
		\item Mais \textbf{en pratique} rares sont les informaticiens qui traitent les données comme des nombres. 
	\end{itemize}
	\position{1}{1}
\end{slide}

\begin{slide}
		\forme{Humanités numériques} : traduction de \forme{digital humanities}.
		\begin{itemize}
			\item Respecte le vocabulaire du français.
			\item Respecte la sémantique de l'anglais.
		\end{itemize}
\position{1}{3}
\end{slide}	
	
\begin{slide}
\forme{Humanités digitales} : transposition de \forme{digital humanities}.
			\begin{itemize}
				\item Ne respecte pas le vocabulaire du français.
				\item Ne respecte pas la sémantique de l'anglais.
				\item \textbf{Mais} \enquote{nous entrons en contact avec le monde digital avec nos doigts} (C. Clivaz). %http://claireclivaz.hypotheses.org/114

				\item \textbf{Néanmoins}: % FIXME: mouais… pour moi digital c'est juste le faux-ami… les distinctions suivantes sont un peu capillotractées
				\begin{itemize}
					\item Concerne non pas l'\textbf{informatique} mais les \textbf{ordinateurs}.
					\item N'est pas nouveau.
					\item Peut s'appliquer à l'étude sociologique de l'utilisation de l'informatique mais difficilement à l'utilisation de l'informatique dans le domaine des SHS.
				\end{itemize}
			\end{itemize}
	\position{1}{3}
\end{slide}
\subsection{Ordinateurs, smartphones et autres machines}


\begin{slide}
  Un ordinateur est une machine qui peut :
  \begin{itemize}
    \item Lire des \textbf{données}. 
    \item Manipuler les données en fonctions d'\textbf{instructions} :
      \begin{itemize}
	\item Modifiables par l'utilisateur.
	\item Non linéaires (possibilité du \textbf{SI / ALORS / SINON}). % FIXME: précise si/alors/sinon… j'ai lu SIcomme l'acronyme de système international
						  % FIXME: tu peux mentionner les structures de contrôle de base: séquence, conditionnelle, boucle, récursion…
      \end{itemize}
    \item Fournir le résultat de ces manipulations. 
  \end{itemize}
  \position{2}{1}
\end{slide}
\begin{slide}
  Techniquement les choix sont multiples:
  \begin{itemize}
    \item Pour la manipulation des données:
      \begin{itemize}
	\item Câbles que l'on déplace et tubes à vides (premiers ordinateurs).
	\item Transistors, aujourd'hui regroupés en micro-processeurs (ordinateurs modernes).
      \end{itemize}
    \item Pour la lecture et l'écriture des données:
      \begin{itemize}
	\item Cartes perforées.
	\item Clavier
	\item Mémoires électroniques (vives et mortes)
	\item Écran (tactiles ou non)
	\item etc.
      \end{itemize}
  \end{itemize}
  
  \position{2}{1}
\end{slide}
\begin{slide}
	 \only{Micro-ordinateurs de bureau, ordinateurs portables, tablettes, smartphones etc.:} % GPS, box internet, lecteur DVD, télé, voiture…
			\begin{itemize}
				\item Sont tous des ordinateurs.
				\item Portent des noms différents pour des raisons commerciales.
				\item Peuvent, pour des raisons juridico-commerciales, être limités dans la possibilité \emph{effective} de modifier les instructions (ex : \textbf{Apple Store}).
			\end{itemize}
	\position{2}{3}
\end{slide}
