\documentclass{beamer}
\usepackage[nobiblatex]{beamer-standrews}
\usepackage{xcolor,amsmath}
\author{Maïeul Rouquette}
\date{1\textsuperscript{er} décembre 2014}
\title{Une introduction au vocabulaire de l'informatique}
\setmainlanguage{french}
\setotherlanguage{english}
\usepackage{minted}
\beamerdefaultoverlayspecification{<+->}
\setbeameroption{hide notes}
\usepackage{graphicx}
\graphicspath{{img/}}

% Diverses commandes utiles pour cet exposé
\newcommand{\forme}[1]{\enquote{#1}}
\newcommand{\formeenglish}[1]{\forme{\textenglish{#1}}}

\definecolor{0}{named}{violet}
\definecolor{1}{named}{brown}
\definecolor{2}{named}{red}
\definecolor{3}{named}{blue}
\newcommand{\puissance}[2]{%
	{\color{#1}%
	#2}%
}

\begin{document}


\begin{slide}
\titlepage

\vfill
\hfill\includegraphics[width=36pt]{cc-by-sa.png}
 \hfill\hbox{}

\end{slide}


\section{Introduction}
\subsection{L'auteur}
\begin{slide}


	\begin{itemize}
		

		\item Actuellement doctorant en littérature apocryphe chrétienne à Lausanne.

		\item (Quasi)-entièrement auto-didacte en informatique:
		
		\begin{description}
			\item[2004] Base du web : HTML/CSS + SPIP.
			\item[2006] Programmation : Python.
			\item[2010] \LaTeX.
		\end{description}
	
	\end{itemize}
\end{slide}

\subsection{Objectifs}

\begin{slide}
	\begin{enumerate}
		\item Éclaircir les concepts de base de l'informatique.
		\item Et ce du point de vue d'un informaticien, pas d'un sociologue ou d'un historien.
	\end{enumerate}
\end{slide}

\section[L'information se manipule]{L'information se manipule sous forme de nombres}

\subsection{Informatique}

\begin{slide}
	\begin{itemize}
	\item Tous les jours nous manipulons de l'information pour accomplir des tâches.
	\item Exemple : établir une liste de course pour la cuisine de la semaine.
		
		\begin{itemize}
			\item Taille du frigo et des espaces de stockages.
			\item Nombre de personnes.
			\item Budget.
			\item Goûts culinaire, adaptés aux circonstances.
			\item etc.
		\end{itemize}
	\item À partir de là on établit une liste (implicite ou explicite) de course
	
	\end{itemize}
\end{slide}

\begin{slide}

Deux idées phares de l'informatique:
	\begin{enumerate}
		\item On peut systématiser le traitement des données.
		\item On peut automatiser le traitement des données.
	\end{enumerate}
\end{slide}

\begin{slide}

	\note[item]{\item De ces deux idées, c'est la première la plus importante, qui définit ce qu'est l'informatique \textbf{scientifique}.}
	
	
	\begin{block}{Informatique}
	Science du traitement systématique de l'information.
	\end{block}
	
	\note[item]{\textbf{En pratique}, l'informatique utilise des machines pour traiter l'information mais:}
	
	\begin{exampleblock}{Citation apocryphe (?) attribuée à Edsger Dijkstra}% Voir https://en.wikiquote.org/wiki/Computer_science#Disputed pour l'attribution
	Computer Science is no more about computers than astronomy is about telescopes.\\
	
	L'informatique n'est pas plus la sciences des ordinateurs que l'astronomie n'est celle des télescopes
	\end{exampleblock}
	
\end{slide}

\begin{slide}
	\begin{block}{Informaticien}<.->
		\begin{itemize}
			\item Conçoit comment structurer des données.
			\item Conçoit comment les manipuler de manière systématique : définit des \textbf{algorithmes}.
		\end{itemize}
	\end{block}
\end{slide}
\subsection{Numérique et digital}

\begin{slide}

	\begin{itemize}
		\item Pour permettre l'automatisation du traitement des données par des machines on les ramène à des \emph{nombres}.
		\item D'où les termes:
			\begin{itemize}
				\item \forme{Numérique}
				\item \forme{Digital} (\formeenglish{digit} = \forme{chiffre})
			\end{itemize}
		\item Mais \textbf{en pratique} rare sont les informaticiens qui traitent les données comme des nombres. 
	\end{itemize}
\end{slide}

\begin{slide}
	\begin{itemize}
		\item \forme{Humanités numériques} : traduction de \forme{digital humanities}.
		\begin{itemize}
			\item Respecte le vocabulaire du français.
			\item Respecte la sémantique de l'anglais.
		\end{itemize}
		
		\item \forme{Humanités digitales} : transposition de \forme{digital humanities}.
			\begin{itemize}
				\item Ne respecte le vocabulaire du français.
				\item Ne respecte pas la sémantique de l'anglais.
				\item \textbf{Mais} \enquote{nous entrons en contact avec le monde digital avec nos doigts} (C. Clivaz). %http://claireclivaz.hypotheses.org/114
				\item \textbf{Néammoins}:
				\begin{itemize}
					\item Concerne non pas l'\textbf{informatique} mais les \textbf{ordinateurs}.
					\item N'est pas nouveau.
					\item Peut s'appliquer à l'étude sociologique de l'utilisation de l'informatique mais difficilement à l'utilisation de l'informatique dans le domaine des SHS.
				\end{itemize}
			\end{itemize}
		\item \textbf{Proposition} : \forme{humanités informatisées} ou \forme{humanité informationalisées}.
	\end{itemize}
\end{slide}
\subsection{Ordinateurs, smartphones et autres machines}

\begin{slide}
	\begin{itemize}
		\item \formeenglish{computer} : \forme{calculateur}.
		\item IBM France invente \forme{ordinateur}, inspiré de \forme{ordonnateur} (désuet).
	\end{itemize}
\end{slide}

\begin{slide}
	\begin{itemize}
		\item Un ordinateur est une réalisation concrète d'une \textbf{machine de Turing universelle} (1936):
			\begin{itemize}
				\item Un ruban divisé en cases contenant de l'information notés sous forme d'alphabet (par ex : 0/1).
				\item Une tête de lecture/écriture :
					\begin{itemize}
						\item Lit ou écrit sur une case à la fois.
						\item Peut se déplacer sur le ruban, vers la gauche ou la droite.
					\end{itemize}
				\item Un registre d'état mémorisant l'état courant de la machine.
				\item Une table d'actions qui détermine :
					\begin{itemize}
						\item Qu'écrire sur la case courante.
						\item Dans quel sens déplacer la tête de lecture.
						\item Comment modifier le registre d'état.
					\end{itemize}
					Et ce en fonction de :
					\begin{itemize}
						\item L'état courant.
						\item Le contenu de la case courante.
					\end{itemize}
				\item Puisque la machine est \textbf{universelle} la table d'action est elle même écrite sur le ruban.
			\end{itemize}
	
		\item C'est donc la réalisation du second postulat de l'informatique : on peut automatiser une série d'opération de transformation de données.
	\end{itemize}
\end{slide}

\begin{slide}
	\begin{itemize}
		\item Plus simplement un ordinateur est une machine qui peut :
			\begin{itemize}
				\item Lire des \textbf{données}. 
				\item Manipuler les données en fonctions d'\textbf{instructions} :
					\begin{itemize}
						\item Modifiables par l'utilisateur.
						\item Non linéaires (possibilité du \textbf{SI}).
					\end{itemize}
				\item Fournir le résultat de ces manipulations. 
			\end{itemize}
		\item Techniquement les choix sont multiples:
			\begin{itemize}
				\item Pour la manipulation des données:
					\begin{itemize}
						\item Cables que l'on déplace et tubes à vides (premiers ordinateurs).
						\item Transitors, aujourd'hui regroupés en micro-processeurs (ordinateurs modernes)
						\
					\end{itemize}
				\item Pour la lecture et l'écriture des données des données:
					\begin{itemize}
						\item Cartes perforées.
						\item Clavier
						\item Mémoire électroniques (vives et mortes)
						\item Écran (tactiles ou non)
						\item etc.
					\end{itemize}
			\end{itemize}
	\end{itemize}
\end{slide}
\begin{slide}
	\begin{itemize}
		\item Micro-ordinateurs de bureau, ordinateurs portables, tablettes, smartphones etc.:
			\begin{itemize}
				\item Sont tous des ordinateurs.
				\item Portent des noms différents pour des raisons commerciales.
				\item Peuvent, pour des raisons juridico-commerciales, être limités dans la possibilité \emph{effective} de modifier les instructions (ex : \textbf{Apple Store}).
			\end{itemize}
	\end{itemize}
\end{slide}
\section{Décrire les données}
\subsection{Des nombres}

\begin{slide}
	\begin{itemize}
		\item Aujourd'hui les données que manipulent la machine sont \textbf{des nombres}
		\item Manipulées sous forme \textbf{binaire} pour des contraintes matérielless:
			\begin{itemize}
				\item Le courant passe ou ne passe pas.
				\item Le métal est chargé ou non chargé.
				\item Le plastique est gravé ou non gravé.
			\end{itemize}  
	\end{itemize}
\end{slide}

\begin{slide}
\begin{exampleblock}{Notation décimale}
\begin{align*}
	% Première ligne d'exemple
	\puissance{3}{4}%
	\puissance{2}{3}%
	\puissance{1}{2}%
	\puissance{0}{5} %
	&= 
	\puissance{3}{4000} &+&\  
	\puissance{2}{300} &+&\  
	\puissance{1}{20} &+&\  
	\puissance{0}{5}\\
	%
	% Seconde ligne d'exemple
	\only<+->{%
		&=
		\puissance{3}{4 \times 1000}
		&+&\  
		\puissance{2}{3 \times 100} 
		&+&\  
		\puissance{1}{2 \times 10} 
		&+&\  
		\puissance{0}{5 \times 1}
		} 
		\\
	%
	% Troisième ligne d'exemple
    \only<+->{%
    	&= 
		\puissance{3}{4 \times 10^3} 
		&+&\  
		\puissance{2}{3 \times 10^2} 
		&+&\  
		\puissance{1}{2 \times 10^1} 
		&+&\  
		\puissance{0}{5 \times 10^0}
		}
		\\
\end{align*}

\end{exampleblock}

\begin{exampleblock}{Notation binaire}
\begin{align*}
	% Première ligne d'exemple
	\puissance{3}{1}%
	\puissance{2}{0}%
	\puissance{1}{1}%
	\puissance{0}{0}%
	&\iff 
	\puissance{3}{1 \times 2^3} 
	&+&\  
	\puissance{2}{0 \times 2^2} 
	&+&\  
	\puissance{1}{1 \times 2^1} 
	&+&\  
	\puissance{0}{0 \times 2^0} 
	\\
	%Seconde ligne d'exemple
	\only<+->{%
		&\iff 
		\puissance{3}{1 \times 8} 
		&+&\  
		\puissance{2}{0 \times 4} 
		&+&\  
		\puissance{1}{1 \times 2} 
		&+&\  
		\puissance{0}{0 \times 1}
		}
		\\	
	% Troisième ligne d'exemple
	\only<+->{%
		&\iff 
		\puissance{3}{8} 
		&+&\  
		\puissance{2}{0} 
		&+&\  
		\puissance{1}{2} 
		&+&\  
		\puissance{0}{0}
		}
		\\
	% Quatrième ligne d'exemple																	
	&\only<+->{\iff 10}
\end{align*}
\end{exampleblock}\end{slide}

\begin{slide}
	\begin{itemize}
		\item 1 unité minimale d'information (0/1) = 1 \textbf{bit}.
		\item 8 \textbf{bits} forme 1 \textbf{octet}.
		\item 1000~\textbf{octets} forment 1~\textbf{ko}
		\item … bien que certains logiciels utilisent encore l'ancien usage 1 ko = 1024 octets.
	\end{itemize}
\end{slide}


\begin{slide}
	\begin{itemize}
		\item La \textbf{machine} manipule des \textbf{nombres}.
		\item L'\textbf{informaticien} manipule lui d'autres \textbf{types de données}:
		\begin{itemize}
			\item Nombres.
			\item Chaînes de caractères.
			\item Tableaux de données.
			\item etc. Chaque langage de programmation propose ses propres types de données.
		\end{itemize}
	\end{itemize}
\end{slide}

\subsection{De l'ASCII à l'Unicode}
\begin{slide}
	\begin{itemize}
		\item Manipuler des chaînes de caractères est un des usages courants de l'informatique.
		\item D'autant plus qu'il est plus facile pour un humain d'utiliser des mots que des nombres pour :
			\begin{itemize}
				\item Représenter les données.
				\item Représenter les instructions.
			\end{itemize}
	\end{itemize}
\end{slide}
\begin{slide}
	\begin{itemize}
		\item Définir une liste de caractères : \textbf{jeu de caractères}.
		\item Attribuer un nombre différent à chaque caractère : \textbf{jeu de caractères codés}.
		\item Définir comment on stocke ces caractères : \textbf{codages des caractères}
	\end{itemize}
\end{slide}
\begin{slide}
	\begin{itemize}
		\item \textbf{ASCII} (1963), \formeenglish{American Standard Code for Information Interchange} : codage de caractères:
			\begin{itemize}
				\item Met fin à la multiplication des précédents codages.
				\item Mais ne permet que de noter l'américain et les caractères informatiques de base (pas d'accents)
			\end{itemize}
		\item Multiplication des encodages pour les langues non anglo-saxonnes.
		\item \textbf{Unicode} (1993) : \textbf{jeu de caractères codés}
			\begin{itemize}
				\item Vise à couvrir tous les caractères anciens et modernes, y compris ceux non encore existants.
				\item Peut s'implémenter sous différents \textbf{codages} : UTF-8, UTF-16, UTF-32.
			\end{itemize}
	\end{itemize}
\end{slide}
\subsection{De la structuration}
\begin{frame}
	\begin{itemize}
		\item Acte n°1 de tout informaticien : penser la structure des données.
		\item Exemple : faire une liste des ingrédients d'une recette de cuisine.
	\end{itemize}
\end{frame}

\begin{frame}

\begin{columns}[T]
	\begin{column}{0.25\textwidth}<.->
		\beamerdefaultoverlayspecification{}
		\begin{exampleblock}{En vrac}
		3 kilos de patates ; 5 choux rouges ; radis : une botte
		\end{exampleblock}	
	\end{column}
	
	\begin{column}{0.25\textwidth}
		\beamerdefaultoverlayspecification{}
		\begin{exampleblock}{Ingrédient/Qt.}
			\begin{itemize}
				\item Patates : 3~kg.
				\item Choux rouges : 5.
				\item Radis : 1~botte.
			\end{itemize}
		\end{exampleblock}
	\end{column}
	
	\begin{column}{0.25\textwidth}
		\beamerdefaultoverlayspecification{}
		\begin{exampleblock}{Ing./Unit./Qt.}
			\begin{itemize}
				\item Patates / 3 / kg
				\item Choux rouges / 5 / \~\
				\item Radis / 1 / botte
			\end{itemize}
		\end{exampleblock}
	\end{column}
	\begin{column}{0.25\textwidth}
		\beamerdefaultoverlayspecification{}
		\begin{exampleblock}{Avec deux listes}
			\begin{itemize}
				\item Patates : 3
				\item Choux rouges : 5
				\item Radis : 1
			\end{itemize}
			\hrule width \textwidth
			\begin{itemize}
				\item Patates : kg
				\item Choux rouges : \~\
				\item Radis : botte
			\end{itemize}
		\end{exampleblock}
	\end{column}
\end{columns}
\end{frame}

\subsubsection{Des langages de descriptions}

\begin{frame}
	\begin{itemize}
		\item Une fois conçu la \texbf{structure des données} l'informaticien va pouvoir les \texbf{encoder} (ou les faire encoder…).
		\item Il va pour cela utiliser un \texbf{langage de description}.
		\item Qu'il choisira en fonction de critères tels que:
			\begin{itemize}
				\item Les outils disponibles pour manipuler ces langages.
				\item La verbosité du langage.
				\item La diffusion du langage.
				\item L'inter-opérabilité qu'il souhaite avec d'autres outils.
				\item etc.
				\item Sa propre formation et ses a-prioris.
			\end{itemize}
	\end{itemize}
\end{frame}
\subsubsection{Des gestionnaires de base de données}

\section{Manipuler les données}
\subsection{Encore du binaire}
\subsection{Des langages de haut niveaux}

\section{Communiquer entre ordinateur} % Prendre le modèle TCP/IP en 4 couches, qu'on simplifie en 3 le but n'est pas de faire un cours sur le sujet https://fr.wikipedia.org/wiki/Suite_des_protocoles_Internet
\subsection{Un support physique}% Couche physique , on pourrait subdiviser avec support physique = Tam tam / feu indien et un système de liaison : éviter de frapper au tam tam en même temps
\subsection{Mettre en réseau}% Couche réseau : IP -> internet. On pourrait ajouter tcp/udp pour vérifier qu'il n'y a pas de perte
\subsection{Qu'échange-t-on?}% Des fichiers, des courriers, des fichiers en lien hypertexte. C'est ici que se place le web

\section[Qui contrôle quoi?]{L'informatique est fait par des humains : qui contrôle quoi?}
\subsection{Formats fermés vs formats ouverts}
\subsection{Logiciels propriétaires vs logiciel libre / open sources} % Faire une secionmt 
\end{document}