\documentclass{beamer}
\usepackage[nobiblatex]{beamer-standrews}

\author{Maïeul Rouquette}
\date{1\textsuperscript{er} décembre 2014}
\title{Une introduction au vocabulaire de l'informatique}
\setmainlanguage{french}
\beamerdefaultoverlayspecification{<+->}
\setbeameroption{hide notes}
\usepackage{graphicx}
\graphicspath{{img/}}

\begin{document}


\begin{slide}
\titlepage

\vfill
\hfill\includegraphics[width=36pt]{cc-by-sa.png}
 \hfill\hbox{}

\end{slide}


\section{Introduction}
\subsection{L'auteur}
\begin{slide}


	\begin{itemize}
		

		\item Actuellement doctorant en littérature apocryphe chrétienne à Lausanne.

		\item (Quasi)-entièrement auto-didacte en informatique:
		
		\begin{description}
			\item[2004] Base du web : HTML/CSS + SPIP.
			\item[2006] Programmation : Python.
			\item[2010] \LaTeX.
		\end{description}
	
	\end{itemize}
\end{slide}

\subsection{Objectifs}

\begin{slide}
	\begin{enumerate}
		\item Éclaircir les concepts de base de l'informatique.
		\item Présenter une brève histoire de l'informatique.
	\end{enumerate}
\end{slide}

\section{L'information se manipule sous forme de nombres}

\subsection{Informatique}

\begin{frame}
	\begin{itemize}
	\item Tous les jours nous manipulons de l'information pour accomplir des tâches.
	\item Exemple : établir une liste de course pour la cuisine de la semaine.
		
		\begin{itemize}
			\item Taille du frigo et des espaces de stockages.
			\item Nombre de personnes.
			\item Budget.
			\item Goûts culinaire, adaptés aux circonstances.
			\item etc.
		\end{itemize}
	\item À partir de là on établit une liste (implicite ou explicite) de course
	
	\end{itemize}
\end{frame}

\begin{frame}

Deux idées phares de l'informatique:
	\begin{enumerate}
		\item On peut systématiser le traitement des données.
		\item On peut automatiser le traitement des données.
	\end{enumerate}
\end{frame}
\subsection{Numérique/Digital}
\subsection{Ordinateurs, smartphones et autres machines}

\section{Décrire les données}
\subsection{Des nombres}
\subsection{De l'ASCII à l'UTF-8}
\subsection{De la structuration}
\subsubsection{Des langages de descriptions}
\subsubsection{Des gestionnaires de base de données}

\section{Manipuler les données}
\subsection{Encore du binaire}
\subsection{Des langages de haut niveaux}

\section{Communiquer entre ordinateur} % Prendre le modèle TCP/IP en 4 couches, qu'on simplifie en 3 le but n'est pas de faire un cours sur le sujet https://fr.wikipedia.org/wiki/Suite_des_protocoles_Internet
\subsection{Un support physique}% Couche physique , on pourrait subdiviser avec support physique = Tam tam / feu indien et un système de liaison : éviter de frapper au tam tam en même temps
\subsection{Mettre en réseau}% Couche réseau : IP -> internet. On pourrait ajouter tcp/udp pour vérifier qu'il n'y a pas de perte
\subsection{Qu'échange-t-on?}% Des fichiers, des courriers, des fichiers en lien hypertexte. C'est ici que se place le web

\section{L'informatique est fait par des humains : qui contrôle quoi?}
\subsection{Formats fermés vs formats ouverts}
\subsection{Logiciels propriétaires vs logiciel libre / open sources} % Faire une secionmt 
\end{document}