\documentclass{beamer}

\usepackage[nobiblatex]{beamer-standrews}
\usepackage{xcolor,amsmath}
\author{Maïeul Rouquette}
\date{1\textsuperscript{er} décembre 2014}
\title{Une introduction au vocabulaire de l'informatique}
\setmainlanguage{french}
\setotherlanguage{english}
\usepackage{minted}
\beamerdefaultoverlayspecification{<+->}
\setbeameroption{hide notes}
\usepackage{graphicx}
\usepackage{siunitx}%notammenet pour notation scientifique


\sisetup{
	output-decimal-marker = {,},
	group-separator = {\text{~}},
}
\graphicspath{{img/}}

% pour éviter d'avoir trop d'espace autour des maths
\abovedisplayshortskip=0pt%
\belowdisplayshortskip=0pt
\abovedisplayskip=0pt%
\belowdisplayskip=0pt%

% Diverses commandes utiles pour cet exposé
\newcommand{\forme}[1]{\emph{#1}}
\newcommand{\formeenglish}[1]{\forme{\textenglish{#1}}}

\newcommand{\titretableau}[1]{\textsc{\emph{\textbf{#1}}}}
\newcommand{\titrecolonne}[1]{\textbf{#1}}

% La commande qui permet de tracer ce petit table

% D'abord des commandes préalable
\usepackage{tikz}
\pgfmathsetmacro{\rapport}{sqrt(2)}

\pgfmathsetmacro\xi{0}
\pgfmathsetmacro\xiv{0.75}
\pgfmathsetmacro\xii{(\xiv-\xi)/3}
\pgfmathsetmacro\xiii{\xii*2}

\pgfmathsetmacro\yi{0}
\pgfmathsetmacro\yiv{-\xiv/\rapport}
\pgfmathsetmacro\yii{(\yiv-\xi)/3}
\pgfmathsetmacro\yiii{2*\yii}
% Et là, la commande qui nous intéresse
\newcommand\position[2]{
	\vfill
	\strut\hfill
  \begin{tikzpicture}[scale=2] % Largeur = 1,4 fois hauteur (proportion de feuilles de papier, arrondi)
    \pgfmathsetmacro\x{(#1-1)*\xii-\xi}
    \pgfmathsetmacro\y{(#2-1)*\yii-\yi}
																									% Lignes horizontales
    \draw(\xi,\yi)--(\xiv,\yi);
    \draw(\xi,\yii)--(\xiv,\yii);
    \draw(\xi,\yiii)--(\xiv,\yiii);
    \draw(\xi,\yiv)--(\xiv,\yiv);

																																	% Lignes verticales

    \draw(\xi,\yi)--(\xi,\yiv);
    \draw(\xii,\yi)--(\xii,\yiv);
    \draw(\xiii,\yi)--(\xiii,\yiv);
    \draw(\xiv,\yi)--(\xiv,\yiv);

																																									% Positionnement du cadre

    \draw[fill=black] (\x,\y) rectangle (\xii+\x,\yii+\y);
  \end{tikzpicture}
  \vspace{\baselineskip}
}


\begin{document}


\begin{frame}
\titlepage

\vfill
\hfill\includegraphics[width=36pt]{cc-by-sa.eps}
 \hfill\hbox{}
 \vskip 1ex
\hfill \url{http://geekographie.maieul.net/152} \hfill\hbox{}

\end{frame}



\section{Introduction}
\subsection{L'auteur}
\begin{slide}


	\begin{itemize}
		

		\item Actuellement doctorant en littérature apocryphe chrétienne à Lausanne.

		\item (Quasi)-entièrement auto-didacte en informatique:
		
		\begin{description}
			\item[2004] Base du web : HTML/CSS + SPIP.
			\item[2006] Programmation : Python.
			\item[2010] \LaTeX.
		\end{description}
	
	\end{itemize}
\end{slide}

\subsection{Objectifs}

\begin{slide}
	\begin{enumerate}
		\item Éclaircir les concepts de base de l'informatique.
		\item Et ce du point de vue d'un informaticien, pas d'un sociologue ou d'un historien.
	\end{enumerate}
\end{slide}

\section[Traiter]{Traiter l'information}

\subsection{Informatique}

\begin{slide}
	\begin{itemize}
	\item Tous les jours nous manipulons de l'information pour accomplir des tâches.
	\item Exemple : établir une liste de courses pour la cuisine de la semaine.
		
		\begin{itemize}
			\item Taille du frigo et des espaces de stockages.
			\item Nombre de personnes.
			\item Budget.
			\item Goûts culinaires, adaptés aux circonstances.
			\item etc.
		\end{itemize}
	\item À partir de là on établit une liste (implicite ou explicite) de courses
	
	\end{itemize}
\end{slide}

\begin{slide}

Deux postulats de l'informatique:
	\begin{enumerate}
		\item On peut systématiser le traitement des données.
		\item On peut automatiser le traitement des données.
	\end{enumerate}
\end{slide}

\begin{slide}

	\note[item]{\item De ces deux idées, c'est la première la plus importante, qui définit ce qu'est l'informatique \textbf{scientifique}.}
	
	
	\begin{block}{Informatique}
	Science du traitement systématique de l'information.
	\end{block}
	
	\note[item]{\textbf{En pratique}, l'informatique utilise des machines pour traiter l'information mais:}
	
	\begin{exampleblock}{Citation apocryphe (?) attribuée à Edsger Dijkstra}% Voir https://en.wikiquote.org/wiki/Computer_science#Disputed pour l'attribution
	\centering
        Computer Science is no more about computers \\
        than astronomy is about telescopes.\\
	
        \medskip
	L'informatique n'est pas plus la science des ordinateurs que l'astronomie n'est celle des télescopes.
	\end{exampleblock}
	
\end{slide}

\begin{slide}
	\begin{block}{Informaticien (sens strict)}<.->
		\begin{itemize}
			\item Conçoit comment structurer des données.
			\item Conçoit comment les manipuler de manière systématique : définit des \textbf{algorithmes}.
		\end{itemize}
	\end{block}
\end{slide}
\subsection{Numérique et \enquote{digital}}

\begin{slide}

	\begin{itemize}
		\item Pour permettre l'automatisation du traitement des données par des machines on les ramène à des \textbf{nombres}.
		\item D'où :
			\begin{itemize}
				\item \forme{Numérique}
				\item \forme{Digital} (\formeenglish{digit} = \forme{chiffre})
			\end{itemize}
		\item Mais \textbf{en pratique} rare sont les informaticiens qui traitent les données comme des nombres. 
	\end{itemize}
\end{slide}

\begin{slide}
	\begin{itemize}
		\item \forme{Humanités numériques} : traduction de \forme{digital humanities}.
		\begin{itemize}
			\item Respecte le vocabulaire du français.
			\item Respecte la sémantique de l'anglais.
		\end{itemize}
		
		\item \forme{Humanités digitales} : transposition de \forme{digital humanities}.
			\begin{itemize}
				\item Ne respecte pas le vocabulaire du français.
				\item Ne respecte pas la sémantique de l'anglais.
				\item \textbf{Mais} \enquote{nous entrons en contact avec le monde digital avec nos doigts} (C. Clivaz). %http://claireclivaz.hypotheses.org/114

				\item \textbf{Néanmoins}: % FIXME: mouais… pour moi digital c'est juste le faux-ami… les distinctions suivantes sont un peu capillotractées
				\begin{itemize}
					\item Concerne non pas l'\textbf{informatique} mais les \textbf{ordinateurs}.
					\item N'est pas nouveau.
					\item Peut s'appliquer à l'étude sociologique de l'utilisation de l'informatique mais difficilement à l'utilisation de l'informatique dans le domaine des SHS.
				\end{itemize}
			\end{itemize}
	\end{itemize}
\end{slide}
\begin{slide}
	\begin{itemize}
		\item \textbf{Proposition} :\\ \forme{humanités informatisées} ou \forme{humanités informationalisées}.
		\item Permet de souligner le traitement systématique et automatique de l'information.
	\end{itemize}
\end{slide}
\subsection{Ordinateurs, smartphones et autres machines}

\begin{slide}
	\begin{itemize}
		\item \formeenglish{Computer}: \forme{calculateur}.
		\item IBM France invente \forme{ordinateur}, inspiré de \forme{ordonnateur} (désuet).
	\end{itemize}
\end{slide}

\begin{slide}
	\begin{itemize}
		\item Un ordinateur est une réalisation concrète d'une \textbf{machine de Turing universelle} (1936):
                        % FIXME: hmm non… la machine de Turing est un modèle mathématique, un outil de preuve, qui sert à exprimer et comprendre la puissance expressive d'un ordinateur. Les ordinateurs n'en sont pas une réalisation (ou alors c'est tellement indirect que ça ne veut plus dire grand chose). Par exemple, les systèmes suivant l'architecture Harvard stockent programme et données dans deux mémoires différentes (par opposition à l'architecture Von Neumann). Il y a d'autres architectures, éventuellement on peut mettre la machine de Turing dedans comme exemple ridiculement inefficace, mais c'est tout.
			\begin{itemize}
				\item Un ruban divisé en cases contenant de l'information notés sous forme d'alphabet (par ex : 0/1).
				\item Une tête de lecture/écriture :
					\begin{itemize}
						\item Lit ou écrit sur une case à la fois.
						\item Peut se déplacer sur le ruban, vers la gauche ou la droite.
					\end{itemize}
				\item Un registre d'état mémorisant l'état courant de la machine.
				\item Une table d'actions qui détermine :
					\begin{itemize}
						\item Qu'écrire sur la case courante.
						\item Dans quel sens déplacer la tête de lecture.
						\item Comment modifier le registre d'état.
					\end{itemize}
				\item Puisque la machine est \textbf{universelle} la table d'action modifiable : elle est elle-même écrite sur le ruban.
			\end{itemize}
	
		\item C'est donc la réalisation du second postulat de l'informatique : on peut automatiser une série d'opération de transformation de données.
	\end{itemize}
\end{slide}

\begin{slide}
	\begin{itemize}
		\item Plus simplement un ordinateur est une machine qui peut :
			\begin{itemize}
				\item Lire des \textbf{données}. 
				\item Manipuler les données en fonctions d'\textbf{instructions} :
					\begin{itemize}
						\item Modifiables par l'utilisateur.
						\item Non linéaires (possibilité du \textbf{SI / ALORS / SINON}). % FIXME: précise si/alors/sinon… j'ai lu SIcomme l'acronyme de système international
                                                  % FIXME: tu peux mentionner les structures de contrôle de base: séquence, conditionnelle, boucle, récursion…
					\end{itemize}
				\item Fournir le résultat de ces manipulations. 
			\end{itemize}
		\item Techniquement les choix sont multiples:
			\begin{itemize}
				\item Pour la manipulation des données:
					\begin{itemize}
						\item Câbles que l'on déplace et tubes à vides (premiers ordinateurs).
						\item Transistors, aujourd'hui regroupés en micro-processeurs (ordinateurs modernes).
					\end{itemize}
				\item Pour la lecture et l'écriture des données:
					\begin{itemize}
						\item Cartes perforées.
						\item Clavier
						\item Mémoire électroniques (vives et mortes)
						\item Écran (tactiles ou non)
						\item etc.
					\end{itemize}
			\end{itemize}
	\end{itemize}
\end{slide}
\begin{slide}
	 \only{Micro-ordinateurs de bureau, ordinateurs portables, tablettes, smartphones etc.:} % GPS, box internet, lecteur DVD, télé, voiture…
			\begin{itemize}
				\item Sont tous des ordinateurs.
				\item Portent des noms différents pour des raisons commerciales.
				\item Peuvent, pour des raisons juridico-commerciales, être limités dans la possibilité \emph{effective} de modifier les instructions (ex : \textbf{Apple Store}).
			\end{itemize}

\end{slide}

\section{Décrire les données}
\subsection{Des nombres}

\begin{slide}
	\begin{itemize}
		\item Aujourd'hui les données que manipulent la machine sont \textbf{des nombres}
		\item Manipulées sous forme \textbf{binaire} pour des contraintes matérielless:
			\begin{itemize}
				\item Le courant passe ou ne passe pas.
				\item Le métal est chargé ou non chargé.
				\item Le plastique est gravé ou non gravé.
			\end{itemize}  
	\end{itemize}
\end{slide}

\begin{slide}
\begin{exampleblock}{Notation décimale}
\begin{align*}
	% Première ligne d'exemple
	\puissance{3}{4}%
	\puissance{2}{3}%
	\puissance{1}{2}%
	\puissance{0}{5} %
	&= 
	\puissance{3}{4000} &+&\  
	\puissance{2}{300} &+&\  
	\puissance{1}{20} &+&\  
	\puissance{0}{5}\\
	%
	% Seconde ligne d'exemple
	\only<+->{%
		&=
		\puissance{3}{4 \times 1000}
		&+&\  
		\puissance{2}{3 \times 100} 
		&+&\  
		\puissance{1}{2 \times 10} 
		&+&\  
		\puissance{0}{5 \times 1}
		} 
		\\
	%
	% Troisième ligne d'exemple
    \only<+->{%
    	&= 
		\puissance{3}{4 \times 10^3} 
		&+&\  
		\puissance{2}{3 \times 10^2} 
		&+&\  
		\puissance{1}{2 \times 10^1} 
		&+&\  
		\puissance{0}{5 \times 10^0}
		}
		\\
\end{align*}

\end{exampleblock}

\begin{exampleblock}{Notation binaire}
\begin{align*}
	% Première ligne d'exemple
	\puissance{3}{1}%
	\puissance{2}{0}%
	\puissance{1}{1}%
	\puissance{0}{0}%
	&\iff 
	\puissance{3}{1 \times 2^3} 
	&+&\  
	\puissance{2}{0 \times 2^2} 
	&+&\  
	\puissance{1}{1 \times 2^1} 
	&+&\  
	\puissance{0}{0 \times 2^0} 
	\\
	%Seconde ligne d'exemple
	\only<+->{%
		&\iff 
		\puissance{3}{1 \times 8} 
		&+&\  
		\puissance{2}{0 \times 4} 
		&+&\  
		\puissance{1}{1 \times 2} 
		&+&\  
		\puissance{0}{0 \times 1}
		}
		\\	
	% Troisième ligne d'exemple
	\only<+->{%
		&\iff 
		\puissance{3}{8} 
		&+&\  
		\puissance{2}{0} 
		&+&\  
		\puissance{1}{2} 
		&+&\  
		\puissance{0}{0}
		}
		\\
	% Quatrième ligne d'exemple																	
	&\only<+->{\iff 10}
\end{align*}
\end{exampleblock}
\end{slide}

\begin{slide}
	\begin{itemize}
		\item 1 unité minimale d'information (0/1) = 1 \textbf{bit}.
		\item 8 \textbf{bits} forme 1 \textbf{octet}.
		\item 1000~\textbf{octets} forment 1~\textbf{ko}
		\item … bien que certains logiciels utilisent encore l'ancien usage 1 ko = 1024 octets.
	\end{itemize}
\end{slide}


\begin{slide}
	\begin{itemize}
		\item La \textbf{machine} manipule des \textbf{nombres}.
		\item L'\textbf{informaticien} manipule lui d'autres \textbf{types de données}:
		\begin{itemize}
			\item Nombres.
			\item Chaînes de caractères.
			\item Tableaux de données.
			\item etc. Chaque langage de programmation propose ses propres types de données.
		\end{itemize}
	\end{itemize}
\end{slide}

\subsection{De l'ASCII à l'Unicode}
\begin{slide}
	\begin{itemize}
		\item Manipuler des chaînes de caractères est un des usages courants de l'informatique.
		\item D'autant plus qu'il est plus facile pour un humain d'utiliser des mots que des nombres pour :
			\begin{itemize}
				\item Représenter les données.
				\item Représenter les instructions.
			\end{itemize}
	\end{itemize}
\end{slide}
\begin{slide}
	\begin{itemize}
		\item Définir une liste de caractères : \textbf{jeu de caractères}.
		\item Attribuer un nombre différent à chaque caractère : \textbf{jeu de caractères codés}.
		\item Définir comment on stocke ces caractères : \textbf{codages des caractères}
	\end{itemize}
\end{slide}

\begin{slide}
	\begin{itemize}
		\item \textbf{ASCII} (1963), \formeenglish{American Standard Code for Information Interchange} : codage de caractères:
			\begin{itemize}
				\item Met fin à la multiplication des précédents codages.
				\item Mais ne permet que de noter l'américain et les caractères informatiques de base (pas d'accents)
			\end{itemize}
		\item Multiplication des encodages pour les langues non anglo-saxonnes.
		\item \textbf{Unicode} (1993) : \textbf{jeu de caractères codés}
			\begin{itemize}
				\item Vise à couvrir tous les caractères anciens et modernes, y compris ceux non encore existants.
				\item Peut s'implémenter sous différents \textbf{codages} : UTF-8, UTF-16, UTF-32.
			\end{itemize}
	\end{itemize}
\end{slide}

\subsection{De la structuration}
\begin{slide}
	\begin{itemize}
		\item Acte n°1 de tout informaticien : penser la structure des données.
		\item Exemple : faire une liste des ingrédients d'une recette de cuisine.
	\end{itemize}
\end{slide}

\begin{slide}

\begin{columns}[T]
	\begin{column}{0.25\textwidth}<.->
		\beamerdefaultoverlayspecification{}
		\begin{exampleblock}{En vrac}
		3 kilos de patates ; 5 choux rouges ; radis : une botte
		\end{exampleblock}	
	\end{column}
	
	\begin{column}{0.25\textwidth}
		\beamerdefaultoverlayspecification{}
		\begin{exampleblock}{Ingrédient/Qt.}
			\begin{itemize}
				\item Patates : 3~kg.
				\item Choux rouges : 5.
				\item Radis : 1~botte.
			\end{itemize}
		\end{exampleblock}
	\end{column}
	
	\begin{column}{0.25\textwidth}
		\beamerdefaultoverlayspecification{}
		\begin{exampleblock}{Ing./Unit./Qt.}
			\begin{itemize}
				\item Patates / 3 / kg
				\item Choux rouges / 5 / $\varnothing$\
				\item Radis / 1 / botte
			\end{itemize}
		\end{exampleblock}
	\end{column}
	\begin{column}{0.25\textwidth}
		\beamerdefaultoverlayspecification{}
		\begin{exampleblock}{Avec deux listes}
			\begin{itemize}
				\item Patates : 3
				\item Choux rouges : 5
				\item Radis : 1
			\end{itemize}
			\hrule width \textwidth
			\begin{itemize}
				\item Patates : kg
				\item Choux rouges : $\varnothing$\
				\item Radis : botte
			\end{itemize}
		\end{exampleblock}
	\end{column}
\end{columns}
\end{slide}

\subsubsection{Des langages de descriptions}

\begin{slide}
	\begin{itemize}
		\item Une fois conçu la \textbf{structure des données} l'informaticien va pouvoir les \textbf{encoder} (ou les faire encoder…).
		\item Il va pour cela utiliser un \textbf{langage de description}.
		\item Qu'il choisira en fonction de critères tels que:
			\begin{itemize}
				\item Les outils disponibles pour manipuler ces langages.
				\item La verbosité du langage.
				\item La diffusion du langage.
				\item L'inter-opérabilité qu'il souhaite avec d'autres outils.
				\item etc.
				\item Sa propre formation et ses a-prioris.
			\end{itemize}
	\end{itemize}
\end{slide}

\begin{slide}
\begin{columns}[T]
	\begin{column}{0.25\textwidth}<.->
		\beamerdefaultoverlayspecification{}
		\begin{exampleblock}{Directement dans le langage de programmation (ici Python)}
			\inputminted{python}{exemples-structure/python.py}
		\end{exampleblock}
	\end{column}
	\begin{column}{0.25\textwidth}
		\beamerdefaultoverlayspecification{}
		\begin{exampleblock}{En \forme{eXtensible Markup Language} (XML)}
			\inputminted{xml}{exemples-structure/xml.xml}
		\end{exampleblock}
	\end{column}
	\begin{column}{0.25\textwidth}
		\beamerdefaultoverlayspecification{}
		\begin{exampleblock}{En \forme{YAML Ain't Markup Language}}
			\inputminted{yaml}{exemples-structure/yaml.yaml}
		\end{exampleblock}
	\end{column}
	\begin{column}{0.25\textwidth}
		\beamerdefaultoverlayspecification{}
		\begin{exampleblock}{En \forme{JavaScript Object Notation} (JSON)}
			\inputminted{json}{exemples-structure/json.json}
		\end{exampleblock}
	\end{column}
\end{columns}
\end{slide}

\subsubsection{Des bases de données relationelles}
\begin{slide}
	\begin{itemize}
		\item On peut mettre des \textbf{données} dans des \textbf{fichiers} qu'on décrypte ensuite.
		\item Mais on peut aussi \textbf{faire abstraction} des fichiers pour ne se focaliser que sur les \textbf{données} et leurs \textbf{relations}.
	\end{itemize}
\end{slide}

\begin{slide}
	\begin{itemize}
		\item On utilise alors un \textbf{gestionnaire de bases de données}.
		\item Qu'on manipule avec un \textbf{langage spécifique}.
		\item Langage commun et standardisé : \enquote{Structured Query Language} (SQL).
	\end{itemize}
\end{slide}

\begin{slide}
	\begin{itemize}
				\item Langage déclaratif.
				\item Pertinents dans certains contextes notamment quand :
				\begin{itemize}
					\item Les relations entres les données sont simples et peut variantes.
					\item Les relations entre les données ne sont pas principalement hiérarchiques.
					\item On cherche plus à obtenir des données en fonctions de leur relations qu'à effectuer des manipulations dessus.
					\item On connaît les contraintes des relations entre données (unicités…)
					\item On veut chercher facilement la modification de plusieurs données
				\end{itemize}
	\end{itemize}
	
\end{slide}
\begin{slide}
	
	\beamerdefaultoverlayspecification{}
	\scriptsize
	\begin{exampleblock}{Tableaux}
	\only<+->{\begin{tabular}{|l|l|c|}
	\hline 
		\multicolumn{3}{|c|}{\titretableau{Ingredients\_recettes}}\\
	\hline
		\titrecolonne{Recette} & \titrecolonne{Ingredients} &  \titrecolonne{Quantite} 
		\\
	\hline
		Bougli-boulga 	&	Patates 		& 3 \\
		Bougli-boulga	&	Choux rouges 	& 5 \\
		Bougli-boulga	& 	Radis 			& 1 \\
	\hline
	\end{tabular}
	}


	\only<+->{
	\begin{tabular}{|l|l|}
	\hline 
		\multicolumn{2}{|c|}{\titretableau{Unites\_ingredients}}\\
	\hline
		\titrecolonne{Ingredient} 	& \titrecolonne{Unite} \\
	\hline
		Patates					& kg \\
		Choux rouges			& \~\ \\
		Radis					& botte \\
		\hline
	\end{tabular}
	}


	\only<+->{\begin{tabular}{|l|c|}
		\hline \multicolumn{2}{|c|}{\titretableau{Recettes}}\\
	\hline
		\titrecolonne{Nom} 
		& 
		\titrecolonne{Temps} 
		\\
	\hline
	Bougli-boulga & 5 mn\\
	\hline
	\end{tabular}
	}
	\end{exampleblock}
	\only<+->{
	\begin{exampleblock}{Requete}
		\inputminted{sql}{exemple-structure/sql.sql}
	\end{exampleblock}
	}
\end{slide}


\section{Manipuler les données}
\subsection{Encore du binaire}
\subsection{Des langages de haut niveaux}

\section{Communiquer entre ordinateur} % Prendre le modèle TCP/IP en 4 couches, qu'on simplifie en 3 le but n'est pas de faire un cours sur le sujet https://fr.wikipedia.org/wiki/Suite_des_protocoles_Internet
\subsection{Un support physique}% Couche physique , on pourrait subdiviser avec support physique = Tam tam / feu indien et un système de liaison : éviter de frapper au tam tam en même temps
\subsection{Mettre en réseau}% Couche réseau : IP -> internet. On pourrait ajouter tcp/udp pour vérifier qu'il n'y a pas de perte
\subsection{Qu'échange-t-on?}% Des fichiers, des courriers, des fichiers en lien hypertexte. C'est ici que se place le web

\section[Qui contrôle quoi?]{L'informatique est fait par des humains : qui contrôle quoi?}
\subsection{Formats fermés vs formats ouverts}
\subsection{Logiciels propriétaires vs logiciel libre / open sources} % Faire une secionmt 
\end{document}