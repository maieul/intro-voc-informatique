% Diverses commandes utiles pour cet exposé
\newcommand{\forme}[1]{\emph{#1}}
\newcommand{\formeenglish}[1]{\forme{\textenglish{#1}}}

\definecolor{0}{named}{violet}
\definecolor{1}{named}{brown}
\definecolor{2}{named}{red}
\definecolor{3}{named}{blue}
\newcommand{\puissance}[2]{%
	{\color{#1}%
	#2}%
}
\newcommand{\titretableau}[1]{\textsc{\emph{\textbf{#1}}}}
\newcommand{\titrecolonne}[1]{\textbf{#1}}

% La commande qui permet de tracer ce petit table

% D'abord des commandes préalable
\usepackage{tikz}
\pgfmathsetmacro{\rapport}{sqrt(2)}

\pgfmathsetmacro\xi{0}
\pgfmathsetmacro\xiv{0.75}
\pgfmathsetmacro\xii{(\xiv-\xi)/3}
\pgfmathsetmacro\xiii{\xii*2}

\pgfmathsetmacro\yi{0}
\pgfmathsetmacro\yiv{-\xiv/\rapport}
\pgfmathsetmacro\yii{(\yiv-\xi)/3}
\pgfmathsetmacro\yiii{2*\yii}
% Et là, la commande qui nous intéresse
\newcommand\position[2]{
	\vfill
	\strut\hfill
  \begin{tikzpicture}[scale=2] % Largeur = 1,4 fois hauteur (proportion de feuilles de papier, arrondi)
    \pgfmathsetmacro\x{(#1-1)*\xii-\xi}
    \pgfmathsetmacro\y{(#2-1)*\yii-\yi}
																									% Lignes horizontales
    \draw(\xi,\yi)--(\xiv,\yi);
    \draw(\xi,\yii)--(\xiv,\yii);
    \draw(\xi,\yiii)--(\xiv,\yiii);
    \draw(\xi,\yiv)--(\xiv,\yiv);

																																	% Lignes verticales

    \draw(\xi,\yi)--(\xi,\yiv);
    \draw(\xii,\yi)--(\xii,\yiv);
    \draw(\xiii,\yi)--(\xiii,\yiv);
    \draw(\xiv,\yi)--(\xiv,\yiv);

																																									% Positionnement du cadre

    \draw[fill=black] (\x,\y) rectangle (\xii+\x,\yii+\y);
  \end{tikzpicture}
  \vspace{\baselineskip}
}

